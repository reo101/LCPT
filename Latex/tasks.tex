%! TEX program = lualatex
%lualatex:
\documentclass{article}
\usepackage{amsfonts}
\usepackage{amsmath, amssymb}
\usepackage{commath}
\usepackage{xargs}
\usepackage{xstring}
\usepackage{mathtools}
\usepackage{xfp}
\usepackage{tikz}
\usetikzlibrary{backgrounds}

\usepackage{geometry}
\geometry{legalpaper, portrait, left=30mm, right=30mm, top=20mm, bottom=20mm}
\usepackage{nopageno}
\usepackage{fontspec}
\usepackage{polyglossia}
\setmainlanguage{bulgarian}
\setotherlanguage{english}
\newfontfamily\cyrillicfont[Script=Cyrillic]{Linux Libertine O}

\usepackage{graphicx}
\graphicspath{ {./} }

\usepackage{amsmath}
\usetikzlibrary{arrows.meta}

\directlua{
    function takovata(x)
        return (x - 2)/math.sqrt(x*x + 1)
    end
}
\pgfmathdeclarefunction{Takovata}{1}{%
  \edef\pgfmathresult{%
     \directlua{tex.print("" .. takovata(#1))}%
   }%
}

\newcommand*{\limi}[2][\infty]{
    \lim\limits_{#2\to#1}
}
%\newcommand*{\sumi}[1][_{i=0}]{
    %\sum\limits_{#2}^{#1}
%}
\newlength{\leftstackrelawd}
\newlength{\leftstackrelbwd}
\def\leftstackrel#1#2{\settowidth{\leftstackrelawd}%
{${{}^{#1}}$}\settowidth{\leftstackrelbwd}{$#2$}%
\addtolength{\leftstackrelawd}{-\leftstackrelbwd}%
\leavevmode\ifthenelse{\lengthtest{\leftstackrelawd>0pt}}%
{\kern-.5\leftstackrelawd}{}\mathrel{\mathop{#2}\limits^{#1}}}
\newcommand{\leftoverinfo}[2]{
    \leftstackrel{\substack{#1}}{#2}
}
\newcommand{\overinfo}[2]{
    \stackrel{\substack{#1}}{#2}
}
\newcommand{\dx}[1][x]{
    \mathrm{d}#1
}
\newcommand{\br}[1]{
    \left(#1\right)
}
\newcommand{\conv}[2][\infty]{
    \underset{#2\to#1}{\longrightarrow}
}
\newcommand{\smallo}[1]{
    o\left(#1\right)
}
\newcommand{\doubleplus}{
    +\kern-1.3ex+\kern0.8ex
}
\newcommand{\der}[1]{
    {\left(#1\right)}'
}
\newcommand{\dilim}[2]{
    \Biggr|_{#1}^{#2}
}
% \newcommand{\Task}[2]{
%     \setcounter{section}{\fpeval{#1-1}}
%     \setcounter{subsection}{\fpeval{#2-1}}
%     \section{Задача}
% }
\newcommand{\Task}[1]{
    \def\taskargs{#1}
    \StrBefore{\taskargs}{,}[\tasksection]
    \StrBehind{\taskargs}{,}[\tasksubsection]
    \setcounter{section}{\fpeval{\tasksection}}
    \setcounter{subsection}{\fpeval{\tasksubsection-1}}
    \subsection{Задача}
}
\newcommand{\HGamma}{
    \overline{\Gamma}
}
\newcommand{\VGamma}{
    % \mathbb{\Gamma}
    % \Delta
    \mathrm{I}\kern-2pt\Gamma%\kern0.0pt
}
\newcommand{\iif}{
    \Longleftrightarrow
}
\renewcommand{\phi}{
    \varphi
}
%\renewcommand{\thesection}{\hspace{1cm}\arabic{section}.}
%\renewcommand{\thesubsection}{\hspace{2cm}\arabic{section}.\arabic{subsection}.}
\begin{document}

\title{Домашно по λСТД}
\author{Павел Атанасов, ФН №62555}
\date{}
\maketitle

\Task{1,3}

Да се покаже, че \(\forall_{l_{1},l_{2}} len(l_{1} \doubleplus l_{2}) = len(l_{1}) + len(l_{2})\)

Индукция по \(l_{1}\):

База: \(l_{1} = []\):
\begin{align*}
    len([] \doubleplus l_{2}) &\leftoverinfo{1.5.1}{=} len(l_{2}) \\
    &= 0 + len(l_{2}) \\
    &\leftoverinfo{1.4.1}{=} len([]) + len(l_{2})
\end{align*}

ИП: Нека е вярно за \(l_{1} = l\)

Стъпка: Нека \(l_{1} = (n:l)\):
\begin{align*}
    len((n:l) \doubleplus l_{2}) &\leftoverinfo{1.5.2}{=} len(n:(l \doubleplus l_{2})) \\
    &\leftoverinfo{1.4.2}{=} 1 + len(l + l_{2}) \\
    &\leftoverinfo{\text{ИП}}{=} 1 + len(l) + len(l_{2}) \\
    &\leftoverinfo{1.4.2}{=} len((n:l)) + len(l_{2})
\end{align*}

\Task{1,5}

\begin{align*}
    \Gamma &: \mathcal{P}\left(U\right) \to \mathcal{P}\left(U\right) \\
    \VGamma &: \mathcal{P}\left(U\right) \to \mathcal{P}\left(U\right) \to \mathcal{P}\left(U\right) \\
    \VGamma\left(A\right)\left(X\right) &:= A \cup \Gamma\left(X\right) \\
    \HGamma &: \mathcal{P}\left(U\right) \to \mathcal{P}\left(U\right) \\
    \HGamma\left(D\right) &:= \mu_{\VGamma\left(D\right)}
\end{align*}

Нека:

\[
    \phi_{\Delta} := \{X \ |\  \Delta\left(X\right) = X\}
\]

Демек:

\[
    \forall \Delta: \mu_{\Delta} \in \phi_{\Delta}
\]

Горните дефиниции важат и за 1.6

\begin{align*}
    \HGamma \text{is monotonous} &\iif A \subseteq B \implies \HGamma\left(A\right) \subseteq \HGamma\left(B\right) \\
    \mu_{\HGamma} &= ?
\end{align*}

Нека \(A \subseteq B\). Тогава:

Знаем, че \(\VGamma\) е фамилия от монотонни оператори, т.е. \(\forall A: \VGamma\left(A\right)\) е монотонен оператор.

От Кнастер-Тарски(КТ):

\[
    \mu_{\Delta} = \bigcap \{X \subseteq U \ |\  \Delta\left(X\right) \subseteq X\}
\]

Демек, ако \(\exists X: \VGamma\left(A\right)\left(X\right) \subseteq X\), т.е. \(X\) е "слаба неподвижна"/преднеподвижна точка, то тогава \(\mu_{\VGamma\left(A\right)} \subseteq X\), защото \(\mu_{\VGamma\left(A\right)}\) е сечението на всички такива преднеподвижни точки.

Разглеждаме:

\begin{align*}
    \VGamma\left(A\right)\left(\mu_{\VGamma\left(B\right)}\right)
    &= A \cup \Gamma\left(\mu_{\VGamma\left(B\right)}\right) \\
    &\leftoverinfo{A \subseteq B}{\subseteq} B \cup \Gamma\left(\mu_{\VGamma\left(B\right)}\right) \\
    &= \VGamma\left(B\right)\left(\mu_{\VGamma\left(B\right)}\right) \\
    &\leftoverinfo{\mu_{\VGamma\left(B\right)} \text{is a fixpoint}}{=} \mu_{\VGamma\left(B\right)} \\
    \implies \VGamma\left(A\right)\left(\mu_{\VGamma\left(B\right)}\right) &\subseteq \mu_{\VGamma\left(B\right)} \\
    \leftoverinfo{\text{КТ}}{\implies} \mu_{\VGamma\left(A\right)} &\subseteq \mu_{\VGamma\left(B\right)} \\
    \leftoverinfo{def}{\implies} \HGamma\left(A\right) &\subseteq \HGamma\left(B\right)
\end{align*}

Следователно \(A \subseteq B \implies \HGamma\left(A\right) \subseteq \HGamma\left(B\right)\)

Нека сега намерим \(\mu_{\HGamma}\):

Разглеждаме:

% \begin{align*}
%     \VGamma\left(\mu_{\Gamma}\right) &= \mu_{\HGamma\(left\mu_{\Gamma}\right)} \\
%     \HGamma\left(\mu_{\Gamma}\right)(X) &= \mu_{\Gamma} \cup \Gamma\left(X\right) \\
% \end{align*}

Нека \(X \in \phi_{\VGamma\left(\mu_{\Gamma}\right)}\):

\begin{align*}
    \VGamma\left(\mu_{\Gamma}\right)\left(X\right)
    &\leftoverinfo{def}{=} \mu_{\Gamma} \cup \Gamma\left(X\right) \\
    &\leftoverinfo{prefixpoint}{\subseteq} X \\
    \implies \mu_{\Gamma} \cup \Gamma\left(X\right) &\subseteq X \\
    \implies \mu_{\Gamma} &\subseteq X
\end{align*}

Обаче \(\mu_{\Gamma} \in \phi_{\VGamma\left(\mu_{\Gamma}\right)}\), понеже:

\begin{align*}
    \VGamma\left(\mu_{\Gamma}\right)\left(\mu_{\Gamma}\right)
    &= \mu_{\Gamma} \cup \Gamma\left(\mu_{\Gamma}\right) \\
    &\leftoverinfo{fixpoint}{=} \mu_{\Gamma} \cup \mu_{\Gamma} \\
    &\leftoverinfo{set\ theory}{=} \mu_{\Gamma}
\end{align*}

Следователно \(\mu_{\Gamma}\) хем е \(\in \phi_{\VGamma\left(\mu_{\Gamma}\right)}\), хем е \(\subseteq X, \forall X \in \phi_{\VGamma\left(\mu_{\Gamma}\right)}\).

Следователно (по КТ) \(\mu_{\Gamma} = \mu_{\VGamma\left(\mu_{\Gamma}\right)}\).

Разглеждаме (ново) \(X \in \phi_{\HGamma}\):

Ще извършим опит да покажем, че \(\HGamma\left(X\right) = X \implies \mu_{\Gamma} \subseteq X\):

% \begin{align*}
%     X
%     &= \HGamma\left(X\right) \\
%     &\leftoverinfo{def}{=} \mu_{\VGamma\left(X\right)} \\
%     &\leftoverinfo{\exists Y\\def}{=} X \cup \Gamma\left(Y\right) \\
%     &\leftoverinfo{fixpoint} = Y \\
%     &\implies X = Y \\
%     &\implies \Gamma\left(Y\right) \subseteq Y \\
%     &\leftoverinfo{\text{КТ}}{\implies} \mu_{\Gamma} \subseteq Y \\
%     &\leftoverinfo{Y = X}{\implies} \mu_{\Gamma} \subseteq X
% \end{align*}

\begin{align*}
    X
    &= \HGamma\left(X\right) \\
    &\leftoverinfo{def}{=} \mu_{\VGamma\left(X\right)} \\
    &\leftoverinfo{def\\fixpoint}{=} X \cup \Gamma\left(X\right) \\
    \implies \Gamma\left(X\right) &\subseteq X \\
    \leftoverinfo{\text{KT}}{\implies} \mu_{\Gamma} &\subseteq X
\end{align*}

Следователно \(\mu_{\Gamma}\) хем е \(\in \phi_{\HGamma}\), хем е \(\subseteq X, \forall X \in \phi_{\HGamma}\).

Следователно (по КТ) \(\mu_{\Gamma} = \mu_{\HGamma}\).

\Task{1,6}

\begin{align*}
    \HGamma\left(\HGamma\left(D\right)\right) &\leftoverinfo{?}{=} \HGamma\left(D\right) \\
\end{align*}

Разглеждаме с цел \(\HGamma\left(D\right) \subseteq \HGamma\left(\HGamma\left(D\right)\right)\):

\begin{align*}
    \VGamma\left(D\right)(\HGamma\left(D\right))
    &:= D \cup \Gamma(\HGamma\left(D\right)) \\
    &\leftoverinfo{set\ theory}{\implies} D \subseteq \VGamma\left(D\right)(\HGamma\left(D\right)) \overinfo{fixpoint}{=} \HGamma\left(D\right) \\
    &\implies D \subseteq \HGamma\left(D\right) \\
    &\leftoverinfo{\HGamma\ monotonous}{\implies} \HGamma\left(D\right) \subseteq \HGamma\left(\HGamma\left(D\right)\right)
\end{align*}

Разглеждаме с цел \(\HGamma\left(\HGamma\left(D\right)\right) \subseteq \HGamma\left(D\right)\):

\begin{align*}
    \VGamma\left(\HGamma\left(D\right)\right)\left(\HGamma\left(D\right)\right) &\overinfo{def}{=} \HGamma\left(D\right) \cup \Gamma\left(\HGamma\left(D\right)\right) \\
    \text{Обаче:\ } \Gamma\left(\HGamma\left(D\right)\right) &\subseteq D \cup \Gamma\left(\HGamma\left(D\right)\right) \\
    &\leftoverinfo{def}{=} \VGamma\left(D\right)\left(\HGamma\left(D\right)\right) \\
    &\leftoverinfo{fixpoint}{=} \HGamma\left(D\right) \\
    \implies \HGamma\left(D\right) \cup \Gamma\left(\HGamma\left(D\right)\right) &\overinfo{\Gamma\left(\HGamma\left(D\right)\right)\subseteq\HGamma\left(D\right)}{=} \HGamma\left(D\right) \\
    \implies \VGamma\left(\HGamma\left(D\right)\right)\left(\HGamma\left(D\right)\right) &= \HGamma\left(D\right) \\
    \leftoverinfo{def}{\implies} \HGamma\left(D\right) &\in \phi_{\VGamma\left(\HGamma\left(D\right)\right)} \\
    \leftoverinfo{\text{КТ}}{\implies} \mu_{\VGamma\left(\HGamma\left(D\right)\right)} &\subseteq \HGamma\left(D\right) \\
    \leftoverinfo{def}{=} \HGamma\left(\HGamma\left(D\right)\right) &\subseteq \HGamma\left(D\right)
\end{align*}

\Task{2,12}

Да се докаже, че релацията \(\leq\) е частична наредба, т.е. че е рефлексивна, транзитивна и антисиметрична. Упътване: покажете, че \(M \leq N \Longleftrightarrow Sub\left(M\right) \subseteq Sub\left(N\right)\)\dots

"\(\implies\)" \(M \leq N\): Индукция по \(N\)
\begin{enumerate}
    \item \(N \equiv x \in V\):
        \begin{align*}
            M \leq x &\implies M \in Sub\left(x\right) = \{x\} \\
            &\implies Sub\left(M\right) = Sub\left(N\right)
        \end{align*}
    \item \(N \equiv N_{1}N_{2}\):
        \(M \leq N_{1}N_{2} \implies M \in Sub(N_{1}N_{2}) = \{N_{1}N_{2}\} \cup Sub(N_{1)} \cup Sub(N_{2})\)
        \begin{enumerate}
            \item \(M \equiv N_{1}N_{2} \implies Sub\left(M\right) = Sub\left(N\right)\)
            \item \(M \in Sub(N_{1}) \overinfo{ИП}{\implies} Sub\left(M\right) \subseteq Sub(N_{1}) \subseteq Sub\left(N\right)\)
            \item \(M \in Sub(N_{2}) \overinfo{ИП}{\implies} Sub\left(M\right) \subseteq Sub(N_{2}) \subseteq Sub\left(N\right)\)
        \end{enumerate}
    \item \(N \equiv \lambda_{x}N_{1}\):
        \(M \leq \lambda_{x}N_{1} \implies M \in \{\lambda_{x}N_{1}\} \cup Sub(N_{1})\)
        \begin{enumerate}
            \item \(M \equiv \lambda_{x}N_{1} \implies Sub\left(M\right) = Sub\left(N\right)\)
            \item \(M \in Sub(N_{1}) \overinfo{ИП}{\implies} Sub\left(M\right) \subseteq Sub(N_{1}) \subseteq Sub\left(N\right)\)
        \end{enumerate}
\end{enumerate}

"\(\impliedby\)" \(Sub\left(M\right) \subseteq Sub\left(N\right)\): Индукция по \(M\)
\begin{align*}
    M \in Sub\left(M\right) &\overinfo{\subseteq}{\implies} M \in Sub\left(N\right) \\
    &\overinfo{def}{\implies} M \leq N
\end{align*}

\end{document}
